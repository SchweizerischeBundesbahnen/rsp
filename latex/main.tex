\documentclass{article}
\usepackage[utf8]{inputenc}
\usepackage{url}
\usepackage{hyperref}
\usepackage{amsmath}
\usepackage{amssymb}
\usepackage{mathrsfs}
\usepackage{mathtools}
\usepackage{stmaryrd}
\usepackage[square,sort,comma,numbers]{natbib}
\usepackage{graphicx}
\usepackage{float}
\usepackage[framemethod=TikZ]{mdframed}
\usepackage{footmisc}
\usepackage{algorithm}
\usepackage{algpseudocode}
\renewcommand{\algorithmicrequire}{\textbf{Input:}}
\renewcommand{\algorithmicensure}{\textbf{Output:}}


\DeclareFontFamily{OT1}{pzc}{}
\DeclareFontShape{OT1}{pzc}{m}{it}{<-> s * [1.10] pzcmi7t}{}
\DeclareMathAlphabet{\mathpzc}{OT1}{pzc}{m}{it}




\title{Real Time Large Railway Network Re-Scheduling}
\author{Erik Nygren\footnote{\url{erik.nygren@sbb.ch}}, Christian Eichenberger\footnote{\url{christian.markus.eichenberger@sbb.ch}}, Emma Frejinger\footnote{\url{emma.frejinger@cirrelt.ca}}}

\date{\today}

\newcommand*{\NNN}[0]{\mathbb{N}}%
\newcommand*\xor{\mathbin{\oplus}}
\newcommand*\GG{\mathcal{G}}
\DeclareMathOperator{\dom}{dom}
\DeclareMathOperator*{\argmax}{arg\,max}
\DeclareMathOperator*{\argmin}{arg\,min}
\DeclarePairedDelimiter{\ceil}{\lceil}{\rceil}
\DeclarePairedDelimiter{\floor}{\lfloor}{\rfloor}
\DeclarePairedDelimiterX{\Set}[1]\{\}{%
  \, #1 \,
}

\begin{document}

\maketitle

\tableofcontents
\begin{abstract}
In this paper, we describe our first steps to tackle the Re-Scheduling Problem for Real-Time Large Railway Networks by a combination of techniques from operations research and of different heuristics from ML or of domain-specific heuristics.

The Industry State of the Art manages to resolve re-scheduling conflicts fully automatically only at narrowly defined hubs. In our approach, we aim at combining the best of two worlds: the rigor of the OR formulation with condensed experience in the form of a hypothesized Oracle. The Oracle's prediction could either narrow down the solution space (hard constraints) or speed up the solution process by strong priorities (soft constraints and solver heuristics).

We believe that the very nature of railway systems allows for very strong heuristics which could allow for the problem to become tractable for large networks in real-time scenarios.

The goal of this paper is four-fold:
\begin{description}
\item[G1] report the decompositional problem formulation in a formal way;
\item[G2] show the validity of the approach for one OR solver;
\item[G3] report our first steps in tackling the Oracle;
\item[G4] provide an extensible playground open-source implementation \url{https://github.com/SchweizerischeBundesbahnen/RSP} for further research.
\end{description}
 We hope that this will draw the attention of both academic and industrial researchers to find other and better approaches and collaboration across Railway companies and from different research traditions.
\end{abstract}



%%%%%%%%%%%%%%%%%%%%%%%%%%%%%%%%%%%%%%%%%%%%%%%%%%%%%%%%%%%%%%%%%%%%%%%%%%
%%%%%%%%%%%%%%%%%%%%%%%%%%%%%%%%%%%%%%%%%%%%%%%%%%%%%%%%%%%%%%%%%%%%%%%%%%
\section{Context and Goals}
%%%%%%%%%%%%%%%%%%%%%%%%%%%%%%%%%%%%%%%%%%%%%%%%%%%%%%%%%%%%%%%%%%%%%%%%%%
%%%%%%%%%%%%%%%%%%%%%%%%%%%%%%%%%%%%%%%%%%%%%%%%%%%%%%%%%%%%%%%%%%%%%%%%%%

%%%%%%%%%%%%%%%%%%%%%%%%%%%%%%%%%%%%%%%%%%%%%%%%%%%%%%%%%%%%%%%%%%%%%%%%%%
\subsection{Real-world Context}
%%%%%%%%%%%%%%%%%%%%%%%%%%%%%%%%%%%%%%%%%%%%%%%%%%%%%%%%%%%%%%%%%%%%%%%%%%
% real-world operations: how is it solved today in the real world, by humans and algorithmic
Switzerland has a dense railway network with both freight and passenger trains running on the same infrastructure. More than 1.2 million people use trains on a daily basis \cite{rcsbrochure}.
In Railway Operations, the operational schedule has to be continually re-computed because of many smaller and larger delays that arise during operations. Not all of those can be absorbed by extra times in the schedule, and if the delay has an impact on other trains, decisions on re-ordering or re-routing trains have to be taken to derive a new feasible operational plan. The industry state of the art is that delay propagation is efficiently re-computed by online IT systems. Conflicts, however, have to be most often resolved by humans by explicitly deciding on re-ordering or re-routing based on their experience. Because of the massive combinatorial complexity of these microscopic models, Operations Research models are currently only applied in very restricted, highly condensed geographic areas for re-ordering decisions but do not consider routing alternatives.



This situation is depicted in a schematic way in Figure~\ref{fig:introduction_compensation}:
%
\begin{figure}[hbtp]
	\centering
  \includegraphics[width=0.9\textwidth]{introduction_compensation.png}
	\caption{Real-time rescheduling has so far been applied to restricted geographic areas only.}
	\label{fig:introduction_compensation}
\end{figure}
%
There is a common view of the current situation within system boundaries; the whole system is decomposed into disjoint geographic cells of responsibility. Most of them are handled by humans: human dispatchers have a view of the full system and can communicate through structured (e.g. IT system of incident messages) or informal ways (e.g. phone call with station managers or locomotive drivers). There are only a few condensation areas \cite{caimi2009} (in bottleneck areas such as merging/approach areas in front of large stations or tunnels) that are operated through automatic systems. Between these areas, trains need to be able to compensate: if trains are reordered, these decisions must be taken into account in the neighboring areas.

Currently, optimization algorithms are only applied to restricted condensation zones \cite{caimi2009}. The decisions made within these condensation zones are taken locally within these geographical areas; in case of conflicts with decisions made outside these areas, human dispatchers will need to compensate:
%
\begin{quote}
    [...], a network separation approach is applied to divide the railway network into zones of manageable size by taking account of the network properties, distinguishing condensation and compensation zones. Condensation zones are usually situated near main stations, where the track topology is complex and many different routes exist. As such an area is expected to have a high traffic density, it is also a capacity bottleneck and trains are required to travel through with maximum allowed speed and thus without time reserves. Collisions are avoided by exploiting the various routing possibilities in the station area. Conversely, a compensation zone connects two or more condensation zones and consists of a simpler topology and less traffic density. Here, time reserves should be introduced to improve timetable stability. The choice of an appropriate speed profile is the most important degree of freedom to exploit in these compensation zones. \cite{caimi2009}
\end{quote}

The following Figure~\ref{fig:introduction_operations} shows the re-scheduling control loop adapted from \cite{rcsbrochure,rcswhitepaper}, as well as our simplified control loop for RSP.
%
\begin{figure}[hbtp]
	\centering
%  \includegraphics[width=0.9\textheight,angle=90]{introduction_operations.PNG}
  \includegraphics[width=1.0\textwidth]{introduction_operations.PNG}
	\caption{Re-scheduling loops in reality and our simplification. Boxes represent systems, arrows represent information flow. Dashed boxes and arrows hint at possible extensions to the RSP loop not implemented in this paper.}
	\label{fig:introduction_operations}
\end{figure}
%
Three main areas are involved: \emph{planning}, \emph{dispatching} and \emph{operation} (interlocking and driving assistance). Dispatchers have access to the current situation and a forecast and make decisions on the level of the operational plan, which are then translated either automatically or by dedicated dispatching team members to the interlocking system; some areas are under the control of an automatic optimization system. As time goes on, new elements from the planned schedule will be introduced to the operational schedule and the planned schedule gives also constraints to re-scheduling: passenger trains must not depart earlier than communicated to customers and the service intention may define further hard or soft constraints such as pecuniar penalties for delay or train dropping. The current situation will be reflected in the operational schedule and the forecast.


Our playground implementation is a simplification (see Section~\ref{subsec:playground} below), shown on the right-hand side of Figure~\ref{fig:introduction_operations}: we only consider an operational plan and a single delay and try to re-schedule such that we have a conflict-free plan, again that stays close to the previous operational plan. We hint at a fully automatic re-scheduling loop as follows: information from the current situation needs to be included into the operational plan by passing it to the optimizer via the Oracle; as time moves on, new trains from the planned schedule need to be included into the operational schedule.

In reality, a distinction is made between
\begin{itemize}
    \item operational schedule containing train routes and train ordering
    \item forecast containing passing times within a time horizon
\end{itemize}
%
The forecast outside the optimization areas needs only be conflict-free for the imminent decisions; conflicts in the future are continually resolved by human dispatchers.

As these condensation areas are supposed to grow, we think that the locality of decision-making might become a problem and some sort of coordination might be needed.
%
Therefore, in our RSP approach, we consider how not to take space-local decisions; also we think of operational schedule and forecast as the same, being conflict-free.

%%%%%%%%%%%%%%%%%%%%%%%%%%%%%%%%%%%%%%%%%%%%%%%%%%%%%%%%%%%%%%%%%%%%%%%%%%
\subsection{Research Approach: Decomposition in Space and Time}
%%%%%%%%%%%%%%%%%%%%%%%%%%%%%%%%%%%%%%%%%%%%%%%%%%%%%%%%%%%%%%%%%%%%%%%%%%
Physical railway infrastructure is expensive in building and maintenance \cite{sr40programm}.
Therefore, the existing infrastructure capacity should be exploited as best as possible.
As the number of trains operating increases and condensation areas increase in number and size, it will become increasingly difficult to compensate for decisions taken within condensation zones or to define restrictions that keep the effects on the neighboring compensation zones as predictable as possible.

We will argue that
\begin{description}
\item [H2] it is possible to predict the affected time-space, either from the problem structure or from historic data (see below, Section~\ref{subec:H2}); by this, we would over-achieve goal \textbf{G3}.
\item [H1] such a prediction allows for a speed-up of the OR model (see below, Section~\ref{subec:H1}); by this, we achieve goal \textbf{G2}.
\end{description}



To tackle this problem, our approach is to combine Operations Research and Domain-specific Learning to get the best of both worlds: an "Oracle" is able to predict the "impact" of a delay, with or without a knowledge base learnt by training; we hope the Oracle could predict which trains and which departures are or could be affected by the delay based on past decisions. This piece of information from the Oracle then helps the solver to constrain the search space or at least drive its search more efficiently (driving the branching process).

This approach is shown in Figure~\ref{fig:introduction_time_space}:
%
\begin{figure}[hbtp]
	\centering
  \includegraphics[width=0.9\textwidth]{introduction_time_space.png}
	\caption{The Oracle gives degrees of freedom in time and space.}
	\label{fig:introduction_time_space}
\end{figure}
the Oracle predicts restrictions in time and space, which are passed to the solver.

\subsection{Research Approach: a Synthetic Playground}\label{subsec:playground}
We now give a short introduction to our playground implementation (G4) and its limitation with respect to real-world features.



\begin{description}
\item[Synthetic Infrastructure and Simplified Resource Model] We use the FLATland toolbox \cite{aicrowdFLATland} to generate a grid world infrastructure consisting of 2D square cells; the infrastructure defines the possible movements to the 4 neighbor cells mirroring railway infrastructure (switches etc.)
\item[Synthetic Timetable and Train Dynamics]
\begin{itemize}
    \item Every train has one single source and target, no intermediate stops, as provided by the FLATland toolbox \cite{aicrowdFLATland}.
    \item Every train has a constant speed (which may be different from train to train), as provided by the FLATland toolbox \cite{aicrowdFLATland}.
    \item The schedule (departure and passing times at all cells) is generated such that the sum of travel times of all trains is minimized within a chosen upper bound; this the generated timetable might not have a realistic structure.
    \item There are no time reserves in the schedule, trains cannot catch up.
    \item There is no distinction between published timetable and operational schedule, we only have an operational timetable; in reality, trains must not depart earlier than published.
    \item There are no connections or vehicle tours (turnrounds).
\end{itemize}
\item[Synthetic route alternatives] We use shortest paths (in reality, in particular in the case of disturbances affecting a whole area, we might need a different scheme knowing the parts that cannot be taken); path cycles are not allowed.
\item[Simple Disturbance Model] We simulate one train $a$ being stopped for $d$ discrete time steps at some time $t$ and call this a malfunction $M=(t,d, a)$; in reality, the delay might not be known or only probabilities can be assumed. In reality, update information also comes in batches and we would need to consider multiple delays in the same update interval. This setup is shown in Figure~\ref{fig:introduction_no_loop}.
\end{description}
%
\begin{figure}[hbtp]
	\centering
  \includegraphics[width=0.4\textwidth]{introduction_no_loop.png}
	\caption{Simple non-iterative re-scheduling with single delay.}
	\label{fig:introduction_no_loop}
\end{figure}





\subsubsection{Hypothesis H1}\label{subec:H1}

Hypothesis 1 is a sanity hypothesis: if we do not have a big speed-up with a perfect oracle, the whole approach must be dismissed. If no speed-up can be found we want to identify the reason and document this to gain further insight and adjust our research aim.

Consider Figure~\ref{fig:introduction_H1}:
%
\begin{figure}[hbtp]
	\centering
  \includegraphics[width=0.9\textwidth]{introduction_H1.png}
	\caption{Rationale of hypothesis H1.}
	\label{fig:introduction_H1}
\end{figure}
%
The initial schedule $S0$ and a malfunction $M$ are passed to an OR solver which is given unbounded time to solve the model to optimality; the resulting re-schedule is $S$.

If we compare the initial schedule $S0$ and the re-schedule $S$, we can take the differences ($Delta\_perfect$):

\begin{itemize}
    \item     every time difference at the same node opens up timing flexibility: the nodes are kept fixed, but time is flexible
    \item every node difference opens up routing flexibility: both nodes and times are kept flexible
\end{itemize}
We will describe this in more detail below in Section~\ref{subsubsec:Delta}.


Now we want to see whether the OR solver $f$, given the perfect information coming through the red arrows, allows for a speed-up, i.e. whether the time $t\_{S^\prime}$  needed for the solver with the restriction is much smaller than the time $t\_S$ for the full problem without the restriction

The rationale of this hypothesis is non-general and asymmetric:
%
\begin{itemize}
    \item
    \emph{non-general}: if we show the speed-up for one particular OR solver, this does not mean that the information can be exploited by every other OR solver for speed-up. However, we conjecture that general setup may be applicable to other OR solvers and models, where the exact content of the Oracle's "hint" passed to the solver might differ. We might strengthen this conjecture by comparing with
\item
    \emph{asymmetric}: If we cannot show the speed-up for one particular solver, this does not exclude that the approach might work with a different OR solver and its particular shape of "perfect information". We might need to consider a different OR solver for our exposition.
\end{itemize}



\subsubsection{Hypothesis H2}\label{subec:H2}
Hypothesis 2 asks whether we can build an oracle that provides a considerable speed-up without access to perfect information, only from the current situation.

Consider Figure~\ref{fig:introduction_H2}:
%
\begin{figure}[hbtp]
	\centering
  \includegraphics[width=0.9\textwidth]{introduction_H2.png}
	\caption{Rationale of hypothesis H2.}
	\label{fig:introduction_H2}
\end{figure}
%
Our Oracle $Omega\_realistic$ now has only access to the current information and possibly its knowledge base, not to the perfect re-schedule for the current situation. Can it provide hints to the OR solver that allows for a speed-up $t\_S / t\_S' >> 1$?

We will not build such an $Omega\_realistic$, but report on some first ideas which illustrate the limitation of a geographic decomposition in our synthetic setting in Section~\ref{sec:H2}. We hope this motivates researchers to invest in this task.



%%%%%%%%%%%%%%%%%%%%%%%%%%%%%%%%%%%%%%%%%%%%%%%%%%%%%%%%%%%%%%%%%%%%%%%%%%
%%%%%%%%%%%%%%%%%%%%%%%%%%%%%%%%%%%%%%%%%%%%%%%%%%%%%%%%%%%%%%%%%%%%%%%%%%
\section{The Pipeline for H1}
%%%%%%%%%%%%%%%%%%%%%%%%%%%%%%%%%%%%%%%%%%%%%%%%%%%%%%%%%%%%%%%%%%%%%%%%%%
%%%%%%%%%%%%%%%%%%%%%%%%%%%%%%%%%%%%%%%%%%%%%%%%%%%%%%%%%%%%%%%%%%%%%%%%%%




%%%%%%%%%%%%%%%%%%%%%%%%%%%%%%%%%%%%%%%%%%%%%%%%%%%%%%%%%%%%%%%%%%%%%%%%%%
\subsection{Overview}\label{subsubsec:H1overview}
%%%%%%%%%%%%%%%%%%%%%%%%%%%%%%%%%%%%%%%%%%%%%%%%%%%%%%%%%%%%%%%%%%%%%%%%%%
We now give a more detailed view of the pipeline for hypothesis H1 with respect to Section~\ref{subec:H1}. We refer to Figure~\ref{fig:H1_overview}:
%
\begin{figure}[hbtp]
	\centering
  \includegraphics[width=0.8\textwidth]{H1_overview.png}
	\caption{Pipeline for hypothesis H1 as functional diagram with intermediate data structures at two levels. The top level consists of four stages.}
	\label{fig:H1_overview}
\end{figure}
%
the pipeline decomposes into four top-level stages:
\begin{description}
\item[Experiment Planning] This stage produces experiment parameters in a grid-like search: it takes a set of \emph{parameter ranges} $(l_1,u_1,n_1),(l_2,u_2,n_2)\ldots$ and produces the Cartesian product $R_1 \times R_2 \times \ldots$ of the expanded parameter ranges $R_i=\bigcup_{\ell=0,\ldots,n_i-1} \left\{ l_i + \ell \cdot \frac{u_i-l_i}{n_i}\right\}$ of size $n_1 \cdot n_2 \cdot \ldots$ (we assume $u_i-l_i$ to be a multiple of $n_i$). We will call this roll-out $R_1\times R_2 \times \cdots$ an \emph{experiment agenda} and its elements $(p_1,p_2,\ldots)\in R_1\times R_2 \times\cdots$ \emph{experiment parameters}.
\item[Experiment Setup] consists of \emph{infrastructure generation} for the given experiment parameters, \emph{schedule generation} and \emph{malfunction generation}. We will cover this stage in detail in Sections~\ref{subsubsec:infrastructuregeneration} and \ref{subsubsec:scheduleandmalfunctiongeneration}, respectively.
\item[Experiment Run] corresponds to the content of Figure~\ref{fig:introduction_H1}. We will cover these data structures and functional process steps in Sections~\ref{subsubsec:scheduleproblemdescription} and \ref{subsubsec:Delta}.
\item[Experiment Analysis] This stage produces the plots that help to verify hypothesis H1. We will give more details in Section~\ref{sec:ResultsH1}.
\end{description}



%%%%%%%%%%%%%%%%%%%%%%%%%%%%%%%%%%%%%%%%%%%%%%%%%%%%%%%%%%%%%%%%%%%%%%%%%%
\subsection{Infrastructure generation}\label{subsubsec:infrastructuregeneration}
%%%%%%%%%%%%%%%%%%%%%%%%%%%%%%%%%%%%%%%%%%%%%%%%%%%%%%%%%%%%%%%%%%%%%%%%%%

We now describe our railway infrastructure and how it is generated, mimicking a natural setup.

The infrastructure consists of a grid of cells. Each cell consists of a distinct tile type which defines the movement possibilities of the agent through the cell. There are 8 basic tile types, which describe a rail network. As a general fact in railway network when on navigation choice must be taken at maximum two options are available.
%
Figure~\ref{fig:H1_railway_elements} (top) gives an overview of the eight basic types. These can be rotated in steps of $45^\circ$ and mirrored along the North-South of East-West axis. The bottom row shows the corresponding translation into a graph structure: squares represent cells, black dots represent an entry to pin to the depicted cell and a grey dot represents the pin in the opposite direction into the neighboring cell.
%
\begin{figure}[hbtp]
	\centering
  \includegraphics[width=0.8\textwidth]{H1_railway_elements.png}
	\caption{Eight basic cell types of a FLATland grid (top) and their translation into a directed graph (bottom).}
	\label{fig:H1_railway_elements}
\end{figure}
%
Formally, the \emph{railway infrastructure} is a tuple  $(\mathcal{C}, \mathcal{V}, \mathpzc{c}, \mathcal{E})$, where
\begin{itemize}
    \item $\mathcal{C}$ is a set of cells,
    \item $\mathcal{V}$ is the set of pins by which the cell can be entered (they are positioned to the north, east, south or east of the cell),
    \item $\mathpzc{c}: \mathcal{V} \to \mathcal{C}$ which associates to each ``pin'' the cell it enters (there are at most $4$ pins for every cell),
    \item $\mathcal{E} \subseteq \mathcal{V} \times \mathcal{V}$, the possible directed transitions in the grid;
\end{itemize}
furthermore, we require $(\mathcal{V},\mathcal{E})$ to be an acyclic graph.



\begin{mdframed}
{\bf TODO Erik} describe infrastructure generation: placement of stations and lines (FLATland)
\end{mdframed}


%%%%%%%%%%%%%%%%%%%%%%%%%%%%%%%%%%%%%%%%%%%%%%%%%%%%%%%%%%%%%%%%%%%%%%%%%%
\subsubsection{Schedule and Malfunction Generation}\label{subsubsec:scheduleandmalfunctiongeneration}
%%%%%%%%%%%%%%%%%%%%%%%%%%%%%%%%%%%%%%%%%%%%%%%%%%%%%%%%%%%%%%%%%%%%%%%%%%


A \emph{railway service intention} consists of set $\mathcal{A}$ of trains or agents; each train $a$ has a source $\sigma(a)\in\mathcal{V}$ and a some target in $\tau(a)\subseteq\mathcal{V}$ and a speed $v(a) \in [0,1]$ and a release time $r$, which specifies how long a cell remains blocked after a train has left it.
%
A \emph{schedule} is a pair $(P, A)$ consisting of
\begin{enumerate}
    \item a function $P$ assigning to each train line the path it takes through the network, and
    \item an assignment $A$ of arrival times to each train line at each node on their path.
\end{enumerate}
A path is a sequence of nodes, pair-wise connected by edges. We write $v\in p$ and $(v, v^\prime)\in p$ to denote that node $v$ or edge $(v, v^\prime)$ are contained in path $p=(v_1, . . . , v_n)$, that is, whenever $v=v_i$ for some $1\leq i \leq n$, or this and additionally $v^\prime=v_{i+1}$, respectively.
%
A path $P(a) = (v_1, . . . , v_n)$ for $a= (S, L, e, l, w)\in \mathcal{A}$ has to satisfy
\begin{equation}
v_i \in S \textrm{ for }1\leq i \leq n \label{eq:ASP_1},
\end{equation}
\begin{equation}
(v_j, v_{j+1})\in L \textrm{ for } 1\leq j \leq n-1 \label{eq:ASP_2}
\end{equation}
\begin{equation}
in(v_1) = 0 \textrm{ and } out(v_n) = 0,\label{eq:ASP_3}
\end{equation}
where $in$ and $out$ give the in- and out-degree of a node in graph $(S, L)$, respectively.
Intuitively, conditions (\ref{eq:ASP_1}) and (\ref{eq:ASP_2}) enforce paths to be connected and feasible for the train line in question and Condition (\ref{eq:ASP_3}) ensures that each path is between a possible start and end node.
An assignment $A$ is a partial function $\mathcal{A}\times V\to \mathbf{N}$, where $A(a, v)$ is undefined whenever$v\not\in P(a)$.
%
Now we require schedules to satisfy
\begin{description}
\item[speed] each agent has a constant \emph{minimum running time}, we require
\begin{equation}
    A(a,v')-A(a,v)\geq {v(a)}^{-1}
\end{equation} for all trains $a\in \mathcal{A}$ and all traversed edges $(v,v^\prime) in P(a)$;
\item[resource exclusion] cells must not be used concurrently and not before the release time after the leaving the cell has elapsed, i.e.
\begin{equation}
A(a, v^\prime) +r \leq A(a^\prime, u) \textrm{ or }A(a^\prime, u^\prime) + r\leq A(a, v) \label{eq:resource_exclusion}
\end{equation}
for all ${a, a^\prime} \subseteq \mathcal{A}$, $a\not=a'$,$(v, v^\prime)\in P(a)$,$(u, u^\prime)\in P(a^\prime)$ with $r(v)=r(u)$.
%
Notice that the exit time corresponds to the entry time in the next cell/edge and should not be confused with the release time of the cell, which comes $r$ time steps after the exit time.
\end{description}


With these definitions, we are now equipped to describe how we generate service intentions and schedules.

\begin{mdframed}
{\bf TODO Erik} describe railway service intention generation (placement of agents in FLATland)
\end{mdframed}

For schedule generation, we use the OR solver model from Section~\ref{subsubsec:scheduleproblemdescription} with a different objective function, namely minimizing
\begin{equation}
\sum_{a \in \mathcal{A}, v,v^\prime \in P(a), in(v)=0, out(v)=0} A(a,v^\prime) - A(a,v)
\end{equation}
and restricting
\begin{equation}
\max_{a \in \mathcal{A}, v \in P(a)} A(a,v) \leq U,
\end{equation}
where we use the following heuristic for the upper bound,
\begin{equation}
U=\delta \cdot  \left(w + h + \frac{\left|\mathcal{A}\right|}{\left| S \right|} \right)
\end{equation}
and $\delta=8$,
to find a schedule $S$ for a given service intention. This objective function is intended to mimick a green wave behaviour of real-world train scheduling (schedules are constructed such that there are only planned stops for passenger boarding and alighting); however, this also keeps us from introducing time reserves in the schedule and trains will not be able to catch up in our synthetic world.


Our experiment parameters contain the following parameters for malfunction generation:
\begin{itemize}
    \item $m_{earliest} \in \mathbb{N}$: the earliest time step the malfunction can happen;
    \item $m_{duration} \in \mathbb{N}$: how much the train will be delayed;
    \item $m_{agent} \in \mathcal{A}$: the train that is concerned.
\end{itemize}

Given the schedule, we derive our malfunction $M=(m_{time\_step},m_{duration},m_{agent})$ where
\begin{equation*}
m_{time\_step} = \min \left\{ \min_{v\in P(m_{agent})} A(m_{agent},v) + m_{earliest}, \max_{v \in P(m_{agent},v)} A(m_{agent},v) \right\},
\end{equation*}
which ensures that the agent malfunction happens while the train is running.



%%%%%%%%%%%%%%%%%%%%%%%%%%%%%%%%%%%%%%%%%%%%%%%%%%%%%%%%%%%%%%%%%%%%%%%%%%
\subsubsection{General Train Scheduling Problem}\label{subsubsec:scheduleproblemdescription}
%%%%%%%%%%%%%%%%%%%%%%%%%%%%%%%%%%%%%%%%%%%%%%%%%%%%%%%%%%%%%%%%%%%%%%%%%%

We will introduce the abstract model from \cite{DBLP:journals/corr/abs-2003-08598}, which is more gerneral than we actually need for our pipeline for H1; this will allow us to review the synthetic assumptions of Section~\ref{subsec:playground} in a more formal setting. We already mentioned in the previous Section~\ref{subsubsec:scheduleandmalfunctiongeneration} that we use this more general model with a dedicated objective to generate the input schedule for our pipeline; in the next Section~\ref{subsubsec:Delta}, we will show how this solver model can be used for re-scheduling.


According to \cite{DBLP:journals/corr/abs-2003-08598},  the \emph{(general) train scheduling problem} is formalized as a tuple $(N, \mathcal{A}, C, F)$ having the following components
\begin{itemize}
    \item $N$ stands for the railway network $(V, E, R, m, a, b)$, where
        \begin{itemize}
            \item $(V, E)$ is a directed graph,
            \item $R$ is a set of resources,
            \item $m:E\to\mathbb{N}$ assigns the minimum travel time of an edge,
            \item $a: \mathbb{R}\to 2^E$ associates resources with edges in the railway network, and
            \item $b:R\to \mathbb{N}$ gives the time a resource is blocked after it was accessed by a train line.
        \end{itemize}
    \item $\mathcal{A}$ is a set of train lines to be scheduled on network $N$. Each train in $\mathcal{A}$ is represented as a tuple $(S, L, e, l, w)$, where
        \begin{itemize}
            \item $(S, L)$ is an acyclic subgraph of $(V, E)$,
            \item $e:S \to \mathbb{N}$ gives the earliest time a train may arrive at a node,
            \item $l:S\to \mathbb{N} \cup \left\{\infty\right\}$ gives the latest time a train may arrive at a node, and
            \item $w:L\to \mathbb{N}$ is the time a train has to wait on an edge.
        \end{itemize}
        Note that all functions are total unless specified otherwise.
        \item $C$ contains connections requiring that a certain train line $a^\prime$ must not arrive at node $n^\prime$ before another train line $a $ has arrived at node $n$ for at least $\alpha$ and at most $\omega$ discrete time steps. More precisely, each connection in $C$ is of form $(t,(v, v^\prime), t^\prime,(u, u^\prime), \alpha, \omega, n, n^\prime)$ such that $a= (S, L, e, l, w)\in \mathcal{A}$ and $a^\prime= (S^\prime, L^\prime, e^\prime, l^\prime, w^\prime)\in \mathcal{A}$, $a\not=a^\prime$,$(v, v^\prime)\in L$,$(u, u^\prime)\in L^\prime$,$\left\{\alpha,\omega\right\} \subseteq \mathbb{Z} \cup \left\{\infty,-\infty\right\}$, and either $n=v$ or $n=v^\prime$, as well as, either $n^\prime=u$ or $n^\prime=u^\prime$.
    \item Finally, $F$ contains collision-free resource points for each connection in $C$. We represent it as a family $(F_c)_{c\in C}$. Connections removing collision detection are used to model splitting (or merging) of trains, as well as reusing the whole physical train between two train lines. More importantly, this allows us to alleviate the restriction that subgraphs for train lines are acyclic, as we can use two train lines forming a cycle that are connected via such connections. Refer to \cite{DBLP:journals/corr/abs-2003-08598} for more details.
\end{itemize}
In this setting, edges can be interpreted as geographic sections with a speed profile; however, the model has no reference to the underlying geography (coordinates etc.); also, resources have no location -- they can be interpreted as track sections that need to be reserved, but they may also be gates or sideway tracks that need to be reserved while travelling the edge.


We now show how our simple scheduling model of Section~\ref{subsubsec:scheduleandmalfunctiongeneration} can be embedded into this more general framework. Let $(\mathcal{A},\sigma,\tau,v,r)$ be a service intention in a railway infrastructure $(\mathcal{C}, \mathcal{V}, \mathpzc{c}, \mathcal{E})$.
Then, $(N,\tilde{\mathcal{A}},C,F)$ with
\begin{itemize}
    \item $\tilde{\mathcal{A}}$ consists of a tuple $(S,L,e,l,w)$ for each $a \in \mathcal{A}$ where Let
        \begin{itemize}
            \item $S=\left\{ v: v \in P_a\right\}$
            \item $L=\left\{ (v_1,v_2): (v_1,v_2) \in P_a\right\}$
            \item $e(v)=\min_{p: \sigma(a)\textrm{--}v\textrm{ path}} \left|p\right| \cdot {v(a)}^{-1}$
            \item $l(v)=\max_{p: v\textrm{--}\tau(a)\textrm{ path}} U-\left|p\right| \cdot {v(a)}^{-1}$
            \item $w(e)={v(a)}^{-1}$\mbox{  }\footnote{This does not respect the intended semantics of the general model. Therefore, it would be better to use $S=\left\{ (v,a): v \in P_a\right\}$, $L=\left\{ ((v_1,a),(v_2,a)): (v_1,v_2) \in P_a\right\}$, $w(e)=0$ and $m(((v_1,a),(v_2,a)))={v(a)}^{-1}$.}.
        \end{itemize}
        for a set $P_a$ of $\sigma(a)$--$\tau(a)$ paths in $\mathcal{E}$.
        \item $N=(V,E,R,m,a,b)$ where
        \begin{itemize}
            \item $V=\bigcup_{a \in \mathcal{A}} V_a$
            \item $E=\bigcup_{a \in \mathcal{A}} E_a$
            \item $R=\mathcal{C}$
            \item $m((v_1,v_2,a))=0$
            \item $a(c) = \left\{ e=(v_1,v_2,a): \mathpzc{c}(v_1)=c \right\}$
            \item $b(c)=r$
        \end{itemize}
    \item $C=\emptyset$
    \item $F=\emptyset$
\end{itemize}
is a train scheduling problem. Some remarks on this transformation:
\begin{itemize}
    \item We only have one resource per edge, i.e. we only reserve the train's own track.
    \item The earliest and latest windows are designed such that they represent the earliest possible time the train can reach the vertex, respectively, the latest possible time the train must pass in order to be able to reach the target within the time limit.
\end{itemize}

The time windows given by $e$ and $l$ can be computed by Algorithm~\ref{algo:propagate_earliest} and \ref{algo:propagate_latest}, respectively. They propagate the minimum running forward from an initial set of earliest and backwards from an initial set of latest constraints. An illustration can be found in Figure~\ref{fig:propagate_earliest} and \ref{fig:propagate_latest}, respectively.
The time windows are thus given by
$$e\leftarrow propagate\_earliest(\Set*{e(\sigma(a))=0}, (S,L), \Set*{\sigma(a)}, {v(a)}^{-1})$$ and $$l\leftarrow propagate\_latest(\Set*{e(\tau)=U: \tau \in S, out(\tau)=0}, (S,L), \Set*{\tau \in S: out(\tau)=0},{v(a)}^{-1}).$$
Finally, we can remove nodes $v$ with empty time window $e(v)>l(v)$.





\begin{algorithm}
	\caption{$propagate\_earliest$} \label{algo:propagate_earliest}
	\begin{algorithmic}[1]
	    \Require $e$, $(S,L)$, $F$, $mrt$
	    \Ensure $e$
	    \State $Open \leftarrow F$
	    \For {$v \in Open$}
    	    \For {$v^\prime \in S-F: (v,v^\prime) \in L$}
    			\State $e(v) \leftarrow \min \Set*{e(v^\prime), e(v)+mrt}$
    			\State $Open \leftarrow Open \cup \Set*{v^\prime}$
    		\EndFor
			\State $Open \leftarrow Open - \Set*{v}$
		\EndFor
	\end{algorithmic}
\end{algorithm}

\begin{figure}[hbtp]
	\centering
  \includegraphics[width=0.8\textwidth]{H1_propagate_earliest.png}
	\caption{Illustration of $propagate\_earliest$ for an agent with speed $\frac{1}{2}$.}
	\label{fig:propagate_earliest}
\end{figure}


\begin{algorithm}
	\caption{$propagate\_latest$} \label{algo:propagate_latest}
	\begin{algorithmic}[1]
	    \Require $l$, $(S,L)$, $F$, $mrt$
	    \Ensure $l$
	    \State $Open \leftarrow F$
	    \For {$v \in Open$}
    	    \For {$v^\prime \in S-F: (v^\prime,v) \in L$}
    			\State $l(v) \leftarrow \max \Set*{e(v^\prime), e(v)-mrt}$
    			\State $Open \leftarrow Open \cup \Set*{v^\prime}$
    		\EndFor
			\State $Open \leftarrow Open - \Set*{v}$
		\EndFor
	\end{algorithmic}
\end{algorithm}


\begin{figure}[hbtp]
	\centering
  \includegraphics[width=0.8\textwidth]{H1_propagate_latest.png}
	\caption{Illustration of $propagate\_latest$ for an agent with speed $\frac{1}{2}$ and for upper bound $U=56$.}
	\label{fig:propagate_latest}
\end{figure}



After this discussion of how our simple instances are translated into the more general train scheduling problem, we define solutions:
As above, the \emph{solution to a train scheduling problem} $(N, \mathcal{A}, C, F)$ is a pair $(P,A)$
satisfying conditions (\ref{eq:ASP_1})--(\ref{eq:ASP_3}).
In addition, given path function $P$, an assignment $A$ has to satisfy the conditions in (\ref{eq:ASP_4}) to (\ref{eq:ASP_8}):
\begin{equation}
A(a, v_i)\geq e(v_i)\label{eq:ASP_4}
\end{equation}
\begin{equation}
A(a, v_i)\leq l(v_i)\label{eq:ASP_5}
\end{equation}
\begin{equation}
A(a, v_j) +m((v_j, v_{j+1})) +w((v_j, v_{j+1}))\leq A(a, v_{j+1})\label{eq:ASP_6}
\end{equation}
for all $a= (S, L, e, l, w)\in\mathcal{A}$ and $P(a) = (v_1, . . . , v_n)$ such that $1\leq i\leq n$,$1\leq j\leq n-1$,
either
\begin{equation}
A(a, v^\prime) +b(r) \leq A(a^\prime, u) \textrm{ or }A(a^\prime, u^\prime) +b(r)\leq A(a, v) \label{eq:ASP_7}
\end{equation}
for all $r\in R$, ${a, a^\prime} \subseteq \mathcal{A}$, $a\not=a'$,$(v, v^\prime)\in P(a)$,$(u, u^\prime)\in P(a^\prime)$ with $\left\{(v, v^\prime),(u, u^\prime)\right\} \subseteq a(r)$ whenever for all $(a,(x, x^\prime), a^\prime,(y, y^\prime), \alpha, \omega, n, n^\prime)\in C$ such that $(x, x^\prime)\in P(a)$,$(y, y^\prime)\in P(a^\prime)$, we have $(a,(v, v^\prime), a^\prime,(u, u^\prime), r)\not\in F_c$, and finally
\begin{equation}
\alpha\leq A(t^\prime, n^\prime)-A(t, n)\leq \omega\label{eq:ASP_8}
\end{equation}
for all $(a,(v, v^\prime), a^\prime,(u, u^\prime), \alpha, \omega, n, n^\prime)\in C$ if $(v, v^\prime)\in P(a)$ and $(u, u^\prime)\in P(a^\prime)$.
%
Intuitively, conditions (\ref{eq:ASP_4}), (\ref{eq:ASP_5}) and (\ref{eq:ASP_6}) ensure that a train line arrives at nodes neither too early nor too late and that waiting and traveling times are accounted for. Furthermore, Condition (\ref{eq:ASP_7}) resolves conflicts between two train lines that travel edges sharing a resource, so that one train line can only enter after another has left for a specified time span. This condition does not have to hold if the two trains use a connection that defines a collision-free resource point for the given edges and resource. Finally, Condition (\ref{eq:ASP_8}) ensures that train line $a$ connects to $a^\prime$at node $n$ and $n^\prime$, respectively, within a time interval from $\alpha$ to $\omega$. Note that this is only required if both train lines use the specific edges specified in the connections. Furthermore, note that it is feasible that $n$ and $n^\prime$ are visited but no connection is required since one or both train lines took alternative routes.


It is easy to see that any solution to the (simplified) \emph{railway service intention} is also a solution to the general train scheduling problem after transformation, hence the given transformation is indeed an embedding, carrying solutions to solutions.



\begin{mdframed}
{\bf TODO Christian} should we introduce removal of edges not reached in time here?\\
{\bf TODO Christian} should we introduce notation for sources $sources(V,E)=Set*{v\in V: in(v)=0}$ and sinks $sinks(V,E)=Set*{v\in V: out(v)=0}$?\\
{\bf TODO Christian} make footnote for generalizing to multiple sources.
\end{mdframed}




%%%%%%%%%%%%%%%%%%%%%%%%%%%%%%%%%%%%%%%%%%%%%%%%%%%%%%%%%%%%%%%%%%%%%%%%%%
\subsubsection{Re-scheduling Full and Delta}\label{subsubsec:Delta}
%%%%%%%%%%%%%%%%%%%%%%%%%%%%%%%%%%%%%%%%%%%%%%%%%%%%%%%%%%%%%%%%%%%%%%%%%%


\noindent\textbf{Re-scheduling Full}
\\
We first describe full the re-scheduling problem as a general train scheduling problem. Let $N=(V,E,R,m,a,b)$ be the network of train scheduling problem, $(P,A)$ be a solution to it and let $M=(m_{time\_step},m_{duration},m_{agent})$ be a malfunction.

Informally, the \emph{re-scheduling problem} is the train scheduling problem such that
\begin{itemize}
    \item all decisions up to and including $m_{time\_step}$ are fixed from the solution. If the train is on an edge at $m_{time\_step}$, then it has to use the same edge, reflecting that a train on a straight segment has to continue and the switch position cannot be changed once the train is on the switch;
    \item the train corresponding to $m_{agent}$ is delayed by $m_{duration}$.
\end{itemize}
We here have two options:
\begin{enumerate}
    \item we can constrain the original scheduling up to the malfunction
    \item we can remove everything up to the malfunction and constrain only the last decision before the malfunction
\end{enumerate}
In order to keep the exposition simple and since this is also in the spirit of a re-scheduling loop with moving time horizon, we adopt the second option in the following exposition.\footnote{The implementation follows option 1. Should we adapt the implementation?}



We now describe the transformations $\Delta_0$ for a single train in Algorithm~\ref{algo:Delta0}:
If the train is already done at the malfunction time step, it can be removed from the problem;
if the train has not started yet at the malfunction time step, we keep its start node fixed from $S0$ and do not start earlier than in $S0$ and use $propagate$ (described below) to derive the constraints.




\begin{mdframed}
{\bf TODO Christian} in case 2 (not started yet), we should respect the malfunction delay? Problem here and in source code!
\end{mdframed}




\begin{algorithm}
	\caption{$Delta_0$ for train $a$} \label{algo:Delta0}
	\begin{algorithmic}[1]
		\Require $(S,L)$, $(P,A)$, $M=(m_{time\_step},m_{duration},m_{agent})$, mrt, U, c
	    \Ensure $(S_1,L_1),e,l$
        \If{$\max_{v\in S} A(v) \leq m_{time\_step}$}
            \State $S_1\leftarrow \emptyset$, $L_1 \leftarrow \emptyset$
        \ElsIf{$\min_{v\in S} A(v) > m_{time\_step}$}
            \State $S_1 \leftarrow S$, $L_1 \leftarrow L$,
            \State $v_1 \leftarrow \argmin_{v\in S} A(v)$
            \State $e(v_1) \leftarrow A(v_1)$
            \For{$\tau \in S_1: out(\tau)=0$}
                \State $l(\tau)=U$
            \EndFor
            \State $F_e \leftarrow \Set*{v_1}$
            \State $F_l \leftarrow \Set*{\tau \in S_1: out(\tau)=0}$
            \State $e,l,(S_1,L_1) \leftarrow propagate(e,l,(S_1,L_1),F_e,F_l,\Set*{v_1}, mrt, U, c)$
        \Else
            \State $e,l,(S_1,L_1) \leftarrow Delta\_0\_running((S,L), (P,A), M, mrt, U, c)$
        \EndIf
\end{algorithmic}
\end{algorithm}


Before describing the third case of the malfunction happening while the train is running, we describe \emph{propagate} (Algorithm~\ref{algo:propagate}): We first propagate earliest and latest, then truncate time windows to $c$; we need to propagate latest again since truncation might spoil the semantics of $l(v)$ being the latest possible passing time to reach the target in time. Finally, we remove nodes that are not reachable in time (because of an empty time window) or cannot be reached from one of the nodes $F_v$ that must be visited.

\begin{algorithm}
	\caption{$propagate$} \label{algo:propagate}
	\begin{algorithmic}[1]
	    \Require $e,l,(S,L),F_e,F_l,F_v,mrt,U,c$
	    \Ensure $e,l,(S,L)$
	    \State $e \leftarrow propagate\_earliest(e, (S,L), F_e, mrt)$
		\State $l \leftarrow propagate\_latest(l,(S,L),F_l, mrt)$
		\If{$c < \infty$}
		\For{$v\in S-F_e$}
	        \State $l(v) \leftarrow \min l(v), e(v)+c$
	    \EndFor
	    \State $l \leftarrow propagate\_latest(l,(S,L),F_l, mrt)$
	    \EndIf
		\State $S \leftarrow \Set{v \in S: e(v) \leq l(v)}$, $L \leftarrow \Set*{(v,v^\prime) \in L: v,v^\prime \in S}$
		\For{$v \in F_v$}
		    \State $S \leftarrow \Set{v^\prime \in S: \textrm{there is a }v\textrm{--}v^\prime\textrm{ path or a }v^\prime\textrm{--}v\textrm{ path in }L}$
		    \State $L \leftarrow \Set*{(v,v^\prime) \in L: v,v^\prime \in S}$
		\EndFor
	\end{algorithmic}
\end{algorithm}



We now describe the remaining case of Algorithm~\ref{algo:Delta0}, namely when the malfunction happens while the train is running. We refer to Algorithm~\ref{algo:Delta0running}: we first determine the edge $(v_1,v_2)$ the train is on when the malfunction happens; we then fix the time for $v_1$ as in $S0$ and set the earliest for $v_2$, possibly delayed. Then, we use $propagate$ to tighten the search space. Notice that $propagate$ will here remove the nodes on the scheduled path before $v_1$ (since the forward propagation will not reach these nodes, they will be removed as the time window will be empty).


\begin{mdframed}
{\bf TODO Christian} introduce convention for undefined e/l as infty / -infty instead of explicit loops?
\end{mdframed}

\begin{algorithm}
	\caption{$Delta\_0\_running$ for running train $a$} \label{algo:Delta0running}
	\begin{algorithmic}[1]
		\Require $(S,L)$, $(P,A)$, $M=(m_{time\_step},m_{duration},m_{agent})$, mrt, U, c
	    \Ensure $e,l,(S,L)$
	    \State $(v_1,v_2) \leftarrow (v_1,v_2) \in L$ s.t. $A(v_1)\geq m_{time\_step}$ and $A(v_2)<m_{time\_step}$
		\State $e(v_1)\leftarrow A(v_1)$, $l(v_1) \leftarrow A(v_1)$
        \If{$a$ corresponds to $m_{agent}$}
            \State $e_1(v_2) \leftarrow A(v_1)+mrt+m_{duration}$
        \Else
            \State $e_1(v_2) \leftarrow A(v_1)+mrt$
        \EndIf
        \State $F_e \leftarrow \Set*{v_1,v_2}$, $F_l \leftarrow \Set*{v_1} \cup \Set*{\tau \in S: out(\tau)=0}$
		\For {$v\in S-F_e$}
	        \State $e(v) \leftarrow \infty$
	    \EndFor
        \For{$v\in S_1-F_l$}
	        \State $l(v) \leftarrow -\infty$
	    \EndFor
	    \For{$\tau \in S_1-\Set*{v_1}: out(\tau)=0$}
	        \State $l(\tau) \leftarrow U$
	    \EndFor
		\State $e,l,(S,L) \leftarrow propagate(e,l,(S,L),F_e,F_l, \Set*{v_1,v_2}, mrt, U, c)$
	\end{algorithmic}
\end{algorithm}







\noindent\textbf{Re-scheduling Delta}
\\
Now we define the perfect oracle as outlined above.
Let $(P_{S0},A_{S0})$ be the solution to the original train scheduling problem
and let $N_S=(V_S,E_S,R_S,m_S,a_S,b_S)$ be the network of the re-scheduling problem for an agent
and  let $(P_S,A_S)$ be a solution to the re-scheduling problem. Algorithm~\ref{algo:Deltaperfect} defines the perfect oracle: we keep fixed all times and nodes that are common to $S_0$ and $S$, respectively. Everything that is not reachable in path ($F=\Delta_P$) or time ($F_e=F_l=\Delta_A)$) will be removed by $propagate$.










\begin{algorithm}
	\caption{$Delta_{perfect}$} \label{algo:Deltaperfect}
	\begin{algorithmic}[1]
		\Require $(S,L)$, $(P_{S_0},A_{S_0})$, $(P_S,A_S)$, $M=(m_{time\_step},m_{duration},m_{agent})$, mrt, U, c
	    \Ensure $e,l,(S_1,L_1)$
	    \State $\Delta_A \leftarrow \Set*{v: A_S(v)=A_{S_0}(v)}$
	    \State $\Delta_P \leftarrow \Set*{v: v\in P_S, v\in P_{S_0}}$
	    \State $S_1 \leftarrow \Set*{v: v\in P_{S_0}} \cup \Set*{v: v\in P_{S}}$
	    \State $L_1 \leftarrow \Set*{(v,v^\prime) \in P_S \textrm{ or } (v,v^\prime) \in P_{S_0}: v,v^\prime \in S_1}$
	   \For{$v \in S_1$}
	        \State $e(v) \leftarrow \infty$, $l(v) \leftarrow -\infty$
	    \EndFor
	    \For {$v \in \Delta_A$}
	        \State $l(v)\leftarrow e(v)\leftarrow A_{S_0}(v)$
	    \EndFor
	    \For{$v \in S_1-\Delta_A: out(v)=0$}
	        \State $l(v) \leftarrow U$
	    \EndFor
	    \State $(v_1,v_2) \leftarrow (v_1,v_2) \in L_1$ s.t. $A_{S_0}(v_1)\geq m_{time\_step}$ and $A_{S_0}(v_2)<m_{time\_step}$
	    \If{$v_2 \not \in \Delta_A$}
        \If{$a$ corresponds to $m_{agent}$}
            \State $e_1(v_2) \leftarrow A_{S_0}(v_1)+mrt+m_{duration}$
        \Else
            \State $e_1(v_2) \leftarrow A_{S_0}(v_1)+mrt$
        \EndIf
        \EndIf
	    \State $e,l,(S_1,L_1) \leftarrow propagate(e,l,(S_1,L_1),\Delta_A \cup \Set*{v_2}, \Delta_A, \Delta_P, mrt, U, c)$
	\end{algorithmic}
\end{algorithm}


\noindent\textbf{Objective for Re-scheduling}
\\
We chose to to minimize a linear combination of delay with respect to the initial schedule $S0$ and penalizing diverging route segments (only the first edge of each such segment). This reflects the idea of not re-routing trains without the need of avoiding delay; if re-scheduling is applied in an iterative loop, this should help to avoid ``flickering'' of decisions.
%
Formally, the objective is to minimize over solutions $S=(P_S,A_S)$  the sum
\begin{equation}
\sum_{a \in \mathcal{A}} \delta\left(S(a,\tau(a)) - S0(a,\tau(a))\right) + \rho \cdot \left|\left\{v \in \mathcal{V}(S,a)-\mathcal{V}(S0,a): (v^\prime,v) \in P_{S0}(a) \right\}\right| \label{eq:objectiveresched}
\end{equation}
where the \emph{delay model} is step-wise linear,
\begin{equation}
\delta(t) =
\begin{cases}
    \infty & \textrm{ if } t \geq \delta_{cutoff},\\
    \floor{t / \delta_{step}} \cdot \delta_{penalty}  & \textrm{else},
\end{cases}
\end{equation}
for hyper-parameters $\delta_{step}$, $\delta_{cutoff}, \delta_{penalty}$
and the second term of (\ref{eq:objectiveresched}) penalizes re-routing with respect to the initial schedule S0 by weight $\rho$ for every the first edge deviating from the original schedule $S0$ .


\begin{mdframed}
{\bf TODO Christian} discuss harmonization with FLUX
\end{mdframed}


%%%%%%%%%%%%%%%%%%%%%%%%%%%%%%%%%%%%%%%%%%%%%%%%%%%%%%%%%%%%%%%%%%%%%%%%%%
%%%%%%%%%%%%%%%%%%%%%%%%%%%%%%%%%%%%%%%%%%%%%%%%%%%%%%%%%%%%%%%%%%%%%%%%%%
\section{Ideas H2}\label{sec:H2}
%%%%%%%%%%%%%%%%%%%%%%%%%%%%%%%%%%%%%%%%%%%%%%%%%%%%%%%%%%%%%%%%%%%%%%%%%%
%%%%%%%%%%%%%%%%%%%%%%%%%%%%%%%%%%%%%%%%%%%%%%%%%%%%%%%%%%%%%%%%%%%%%%%%%%
        Heuristic ideas
        ML Ideas

%%%%%%%%%%%%%%%%%%%%%%%%%%%%%%%%%%%%%%%%%%%%%%%%%%%%%%%%%%%%%%%%%%%%%%%%%%
%%%%%%%%%%%%%%%%%%%%%%%%%%%%%%%%%%%%%%%%%%%%%%%%%%%%%%%%%%%%%%%%%%%%%%%%%%
\section{Discussion: show links to}
%%%%%%%%%%%%%%%%%%%%%%%%%%%%%%%%%%%%%%%%%%%%%%%%%%%%%%%%%%%%%%%%%%%%%%%%%%
%%%%%%%%%%%%%%%%%%%%%%%%%%%%%%%%%%%%%%%%%%%%%%%%%%%%%%%%%%%%%%%%%%%%%%%%%%
        ML
        delay propagation / recourse
        simulation
        OR / heuristics
        decomposition approaches
        real-time rescheduling


%%%%%%%%%%%%%%%%%%%%%%%%%%%%%%%%%%%%%%%%%%%%%%%%%%%%%%%%%%%%%%%%%%%%%%%%%%
%%%%%%%%%%%%%%%%%%%%%%%%%%%%%%%%%%%%%%%%%%%%%%%%%%%%%%%%%%%%%%%%%%%%%%%%%%
\section{Results H1}\label{sec:ResultsH1}
%%%%%%%%%%%%%%%%%%%%%%%%%%%%%%%%%%%%%%%%%%%%%%%%%%%%%%%%%%%%%%%%%%%%%%%%%%
%%%%%%%%%%%%%%%%%%%%%%%%%%%%%%%%%%%%%%%%%%%%%%%%%%%%%%%%%%%%%%%%%%%%%%%%%%



%%%%%%%%%%%%%%%%%%%%%%%%%%%%%%%%%%%%%%%%%%%%%%%%%%%%%%%%%%%%%%%%%%%%%%%%%%
%%%%%%%%%%%%%%%%%%%%%%%%%%%%%%%%%%%%%%%%%%%%%%%%%%%%%%%%%%%%%%%%%%%%%%%%%%
\section{Early Results H2}
%%%%%%%%%%%%%%%%%%%%%%%%%%%%%%%%%%%%%%%%%%%%%%%%%%%%%%%%%%%%%%%%%%%%%%%%%%
%%%%%%%%%%%%%%%%%%%%%%%%%%%%%%%%%%%%%%%%%%%%%%%%%%%%%%%%%%%%%%%%%%%%%%%%%%



\bibliographystyle{unsrt}
\bibliography{biblio}

\end{document}






\section{Introduction}
Here we describe the Re-Scheduling problem.

Switzerland has a dense railway network with both freight and passenger trains running on the same infrastructure. More than 1.2 million people use trains on a daily basis. In Railway Operations, the operational schedule has to be continually re-computed because of many smaller and larger delays that arise during operations. Not all of those can be absorbed by extra times in the schedule, and if the delay has an impact on other trains, decisions on re-ordering or re-routing trains have to be taken to derive a new feasible operational plan. The industry state of the art is that delay propagation is efficiently re-computed by online IT systems. Conflicts, however, have to be most often resolved by humans by explicitly deciding on re-ordering or re-routing based on their experience. Because of the massive combinatorial complexity of these microscopic models, Operations Research models are currently only applied in very restricted, highly condensed areas for re-ordering decisions and only very limited routing alternatives, if any.
Research Approach

To tackle this problem, our approach is to combine Operations Research and Machine Learning to get the best of both worlds: a Graph Neural Network acts as an Oracle for Operations Research by learning a heuristic that predicts the "impact" of a delay; we hope the Oracle could predict which trains and which departures are or could be affected by the delay based on past decisions. This piece of information from the oracle then helps the (MILP or CP) solver to constrain the search space or at least drive its search more efficiently (driving the branching process). We hope that Graph Neural Networks are a natural way to represent the graph structure of railway networks.

Our considerations are based on the Re-Scheduling Effects Model of Figure~\ref{fig:EffectsModel}:
the Operational Schedule is based on being informed about delays and the schedule communicated to customers. Interventions may be redefine the allowed speeds per location or per train, the available routes, the connections to be kept or dropped and trains to be added or dropped. Interventions may be directly based on events happening in reality or based on the current network status aggregated from past events.
%
Events from reality are interpreted when lead to a difference to the Operational Schedule.
%
\begin{figure}[hbtp]
	\centering
  \includegraphics[width=0.7\textwidth]{Wirkungsmodell.png}
	\caption{Re-Scheduling Effects Model. Arrows represent logical dependencies. The symbols on the arrows are those used in the formal problem description later on.}
	\label{fig:EffectsModel}
\end{figure}
\section{Related Work}

TODO Related Work. How does it relate to \cite{DBLP:journals/corr/abs-1807-11876}?
TODO Other different approaches: Benders-like decomposition and loop, generate-and-verify-loop


\section{Methodology}
Here we describe the three hypotheses and the pipeline.

\subsection{3 Hypotheses, one building on the previous}
\begin{description}
\item[Hypothesis 1] We can compute good recourse actions, i.e., an adapted plan within a small fraction of the full solution time if all variables are fixed, except those related to services that are affected by the disruption implicitly or explicitly.
\item[Hypothesis 2] Machine learning can predict services that are affected by disruptions implicitly or explicitly with respect to a fixed infrastructure and a fixed basic (published) schedule, without the intention of model training to generalize over infrastructures nor completely new timetable concepts.
\item[Hypothesis 3] If hypothesis 2 is true, in addition, machine learning can predict the state of the system in the next time period after re-scheduling.
\end{description}

\subsection{A General Framework}
Our approach can be seen as a general framework with the following ingredients:
\begin{description}
\item[Detailed Model] We have a model  that reflects the detailed  view but which cannot be applied because it would be intractable. A detailed model acts at the operational and could be produced in reality if no further delays or disruptions happen; it not only works at the more imprecise tactical level. Formally, the detailed model is a deterministic solver that produces an operational plan ${\bf y}^*({\bf x})$ for a problem ${\bf x}$.
\item[Oracle] An Oracle predicts relevant parts of the solution such that the detailed exact model becomes tractable if we complement it by the informed guess by the Oracle.
Formally, an Oracle ${\cal O}$ produces an enhanced input ${\cal O}({\bf x})$ such that ${\bf y}^*({\cal O}({\bf x}))$ can be computed more efficiently.
\item[Simulation] Verifying an Oracle in the real world is often too expensive, either in financial terms or because of legal or other risk issues. Simulation allows us to train and verify the Oracle without having to run those risks. More formally, a simulation produces realistic inputs ${\bf x}$ (following the same distribution).
\end{description}



\subsection{Pipeline for Verification of Hypothesis 1}
Input generation
Step
Generate infrastructure
Generate schedule
Generate disturbance
Full Schedule (w/o disturbance)
Hypothesis Validation

How big is the speed up if we know Delta?
Step
Re-Scheduling full w/o restrictions
Determine Delta w.r. to Full Schedule
Re-Scheduling Delta



\section{Formal Problem Definition}
\subsection{FLATland Formal Definition}
Gridworld and Semantics. I'd like to formalize this. I think it would be worth to describe the FLATland semantics (synchronization, entering Grid, malfunctions, speed model). This will help us to generalize/abstract.


\subsection{Scheduling Problem}
Our formulation is based on \cite{DBLP:conf/lpnmr/AbelsJOSTW19}, but we slightly modify it in order to make the equivalence of microscopic halting sections explicit in the mathematical formulation. We will need this explicit property in the definition the Re-Scheduling Problem.\footnote{TODO: consistency of connections is not enforced - do we need it?} The naming is based on \cite{crowdAISBB}.


A \emph{Scheduling Problem} is a triple $(N,T,C)$ consisting of
\begin{itemize}
    \item a \emph{Railway Network} $N=(V,E,R,m,a,b)$ where
    \begin{itemize}
        \item \emph{global route graph} $(V,E)$, which  is a directed graph; the edges are also called \emph{route sections}
        \item $R$ is a set of resources
        \item \emph{minimum travel times} per route section $m: E \to \NNN$
        \item \emph{resource allocations} on route sections $a: R \to 2^E$
        \item \emph{blocking or release times} per resource $R\to \NNN$

    \end{itemize}
    \item \emph{trains} T of the form $T\ni t = (V^t,E^t,e^t, l^t,w^t)$ where
    \begin{itemize}
        \item \emph{route graph} $(V^t,E^t)$ is an acyclic sub-graph of $(V,E)$
        \item \emph{earliest passing time} of train $t$ at a vertex: $e^t: V^t \to \NNN$
        \item \emph{latest passing time} of train $t$ at a vertex $l^t: V^t \to \NNN$
        \item \emph{waiting times} of train $t$ on edge $w^t: E^t \to \NNN$
        \item a partition $W^t=\{W_1^t,\ldots,W_n^t\} \subseteq 2^{E^t}$ of equivalent microscopic edges such that every path  $p=(v_0,\ldots,v_k) \subseteq V^t$ for $(v_i,v_{i+1})\in E^t$ for $i=1,\ldots,n-1$ \footnote{We use the following notation: $(v_0,\ldots,v_n)\subseteq V^t$ denotes an ordered subset of elements of $V^t$. TODO: Should we formally introduced the set of paths in $V^t$, relating to $E^t$ to simplify notation?} such that $in(v_0)=0$ and $out(v_n)=0$ contains exactly one element of each partitioning element $W_i$, $i=1,...,n$ (i.e. there is exactly one edge $w(W_i^t)=(v_j,v_{j+1})\in E^t$ for $W_i^t \in W$ and  $v_j,v_{j+1}\in p$ and such that all $w(W_i^t)$ are different (i.e. $w$ is injective)).
    \end{itemize}
    \item \emph{connections} $C\subseteq T \times E \times T \times E \times \NNN$ such that $(t_1,e_1,t_2,e_2,c)\in C \implies e_1 \in E^{t_1}, e_2 \in E^{t_2}$
\end{itemize}
Note that $(V,E)$ needs not be connected, i.e. each train may have its own (disjoint) route sub-graph, which is linked to other trains' route graph only through shared resources in $R$, and the union of of all train-specific route graphs needs not cover $(V,E)$.
Notice that the some consistency conditions of the real-world problem are not part of this formal definition, for instance we would have to ensure that every path can produce the commercial stops on different platform tracks.

A solution $(P,A)$ of the Scheduling Problem consists of the
\begin{itemize}
    \item the selected paths $P: T \to 2^V$ where $p(t)=(t_1,\ldots,t_n)\subseteq V^t$ (we use $\ell(p(t))=n$ for the length of the path) is an ordered set satisfying
        \begin{equation}
            (v_i, v_{i+1})\in E^t\textrm{  for }i\in\{1,\ldots,n-1\} \label{eq:path_consistency}
        \end{equation}
        \begin{equation}
            in(v_1)=0\textrm{ and } out(v_n)=0 \label{eq:start_and_target_node}
        \end{equation}
        where $in:V\to\NNN$ and $out:V\to\NNN$ associate in and out degrees of vertices.
    \item partial allocation $A: T\times V \to \NNN$ where $A(t,v)$ is defined for $v\in P(t)$ satisfying
        \begin{equation}
            A(t,v) \geq e^t(v)\textrm { for } v \in P(t) \label{eq:earliest_requirement}
        \end{equation}
        \begin{equation}
             A(t,v) \leq l^t(v)\textrm { for } v \in P(t) \label{eq:latest_requirement}\\
        \end{equation}
        \begin{equation}
             A(t,v_i) + m((v_i,v_{i+1}) + w^t((v_i,v_{i+1})) \leq A(t,v_{i+1}) \label{eq:minimum_running_time_requirement}
        \end{equation}
        \begin{eqnarray}
             A(t_1, v^\prime) +b(r)\leq A(t_2, u) \xor A(t_2, u^\prime) +b(r) \leq A(t_1, v) \nonumber\\
             \textrm { for } v,v^\prime in  \in P(t_1), u,u^\prime \in P(t_2), (v,v^\prime),(u,u^\prime)\in a(r)
             \label{eq:mutual_resource_allocation_requirement}
        \end{eqnarray}
        (where $\xor$ denotes exclusive or)
        \begin{equation}
             A(t_1, v) +c\leq A(t_2, u) \textrm{ for }(t_1,(v,v^\prime),t_2,(u,u^\prime),c) \in C\label{eq:connection_requirement}
        \end{equation}
\end{itemize}

\subsection{FLATland as Scheduling Problem}
How are the route graphs derived from FLATland in general?


\subsection{Commercial and Operational Schedule}
The result of the above (offline) Scheduling Approach is a conflict-free (microscopic) solution. We therefore call it an \emph{Operational Schedule}.
If everything runs smoothly, it contains all the information required to run trains.

What we publish to customers needs not be as details and may not even be conflict-free in the microscopic sense. For instance, we might publish two trains to start at the same time and decide online which one goes first whichever is ready first.
Formally, given an Operational Schedule $(P,A)$,
a \emph{Commercial or Published Schedule} captures the elements published to consumers\footnote{TODO: Should the connections be part of the Commercial Schedule as well?}. Formally,
\begin{itemize}
    \item $S(t,v)$ is the published schedule, partially defined for $v_{i+1} \in P(t)=(v_1,\ldots,v_n)$ where $w^t(v_i)>0$ and satisfying $S(t,v)\leq A(t,v)$ and $S(t,v_j) \leq S(t,v_k)$ for $j<k$ and $v_j,v_k \in P(t)$
\end{itemize}
i.e. we allow for conflicts in the commercial schedule, but guarantee that we never depart earlier than published in $S$.



\subsection{Re-Scheduling Problem}
Given a
\begin{itemize}
    \item Operational Schedule $(P,A)$,
    \item Commercial Schedule $S$
    \item the current time $H \in \NNN$
    \item operational interventions in the form of a scheduling problem $(N^*,T^*, C^*)$
    \item an event $A^\lightning(t^\lightning,v^\lightning)\geq H$, $A^\lightning(t,v)\not=A^(t,v)$, $A(t,v) \geq H$
\end{itemize}
The interpretation of $H$ is that we have received all information up to $H$, i.e. all times  $A(v,t) \leq H$ are fixed, i.e. they are already included included in our current operational schedule (either because they worked according to our initial plan or because we have already updated our current operational schedule). We require $A^\lightning(t,v) \geq H$; the interpretation is that $A^\lightning(t^t\lightning,v^\lightning)$ is a prognosis when train $t$ will pass at $v$ for $A(t,v)\geq H$.
Examples:
\begin{itemize}
    \item If $A^\lightning(t^\lightning,v^\lightning)=H$ and we know that we will be ready only later on at $A^\lightning(t^\lightning,v^\lightning)\not=A^(t^\lightning,v^\lightning)$, $A(t,v) \geq H$.
    \item If $A(t^\lightning,v^\lightning)>H$, we might know that we will be ready earlier at $A^\lightning(t^\lightning,v^\lightning) < A(t^\lightning,v^\lightning)$
\end{itemize}
A \emph{Re-Scheduling Oracle} is a pair $(F_1,F_2)$
\begin{itemize}
    \item $F_1\subseteq T^*$ and $\GG(t)=(\bar{V}^t,\bar{E}^t)$ is an acyclic connected sub-graph of $(V,E)$ for $t \in F_1$
    \item $F_2\subseteq T^*\times V$ where $(t,v)\in F_2 \implies A(t,v)\geq H, v\in P(t), t\not\in F_1$.
    \item  $T(F_2)\cap F_1 = \emptyset$ where $T(F_2) = \{ t: (t,v) \in F_2)$
\end{itemize}
The situation is depicted in Figure~\ref{fig:ReSchedulingFlexi}: there are trains with no flexibility, trains with time flexibility and trains with routing and time flexibility. Notice that the route graphs $\GG(t)$ may grow or shrink with respect to the initial Scheduling Problem, i.e. we may allow for route alternatives that were not allowed for planning. We even allow for trains and connections to be added or dropped.
%
\begin{figure}[hbtp]
	\centering
  \includegraphics[width=0.7\textwidth]{flexibility.png}
	\caption{Degrees of flexibility for the Re-Scheduling-Problem.}
	\label{fig:ReSchedulingFlexi}
\end{figure}



A Solution $(\bar{P},\bar{A})$ of the Re-Scheduling Problem is a solution to the Scheduling Problem $(N^*,T^*,C^*)$, further satisfying
\begin{equation}
    \bar{P}(t) = P(t) \textrm{ for } t \not \in F_1,\label{eq:freedom_alternative_paths}
\end{equation}
\begin{equation}
    \bar{A}(t,v) = A(t,v) \textrm{ for } t \not \in F_1, t \not \in T(F_2), \label{eq:freedom_times_only}
\end{equation}
\begin{equation}
    \bar{A}(t,v^\prime) \geq S(t,v) \textrm { for all } S(t,v) \textrm{ and some } v^\prime \in W(_,v) \label{eq:respect_commercial_schedule}
\end{equation}
as well as equations (\ref{eq:path_consistency}),
(\ref{eq:start_and_target_node}),
(\ref{eq:minimum_running_time_requirement}),
(\ref{eq:mutual_resource_allocation_requirement}),
(\ref{eq:connection_requirement}) above and minimizing
\begin{equation}
    \sum_{(t,v) \in \dom(S)} \bar{A}(t,v) -S(t,v).
\end{equation}
The Re-Scheduling Oracle defines the degrees of freedom on two levels, allowing path alternatives for trains $t\in F_1$ and allowing for modified departure times for the departures in $F_2$, but we require trains to respect the Commercial Schedule, i.e. we do not allow for trains or stops to be dropped.

\section{Results Hypothesis 1}

\subsection{Implementation Details}
k shortest paths, constraints in the model above, best-case Oracle

Our setting for Re-Scheduling for step XX in \cite{DBLP:journals/corr/abs-1807-11876}:
\begin{itemize}
    \item $N^*=N$
    \item $T$ = shortest path or all routes
    \item $C^*=C$
    \item $F_1$, $F_2$?
\end{itemize}{}

\section{Results Hypothesis 2}
\subsection{DL approaches}

\subsubsection{Graph-based Neighborhood Search (GNN)}
Approach Christian B. + Mayra

\subsubsection{Probabilistic Schedule (GNN)}
Erik

\subsubsection{Reinforcement Learning}
Erik with Jakab/Levent?

\subsection{Heuristics Oracle}

\subsubsection{Deadlock Avoidance}
Extending Adrian's Deadlock-Avoidance Algorithm? Christian E. and Adrian?



\subsubsection{Job Insertion}
Blocking Job-Shop-Scheduling Fribourg (Reinhard Bürgy)? Christian E. with Reinhard?

\subsection{Exploiting the Oracle in OR}
Christian E.

\section{Conclusion}
What have we achieved?

\section{Future Work}
Describe the Research Plan and the Challenge for other Researcher.


\section*{To Discuss}
\begin{itemize}
    \item definition of operational schedule should be independent of the scheduling problem based on time windows, definition is not very readable
    \item - not use $t$ for train but for time? Confusing for physicists....
    \item use real numbers for event times
    \item which cost function (classes) do we want to consider?
    \item use different mathematical problem formulation (alternative/disjunctive graph job shop scheduling,....?)
    \item Further Research: not only consider isolated events?
    \item Further Research: drop trains, drop halts, ...
    \item Further Research distributions or time windows as prognosis not only fixed event time
\end{itemize}


\bibliographystyle{unsrt}
\bibliography{biblio}

\end{document}
